%\documentclass[a4paper]{article}
\documentclass{sig-alternate}
\usepackage{amsmath}
\usepackage{amssymb}
\usepackage{graphicx}
\usepackage{subfig}
% \DeclareCaptionType{copyrightbox}
\usepackage{cite}
\usepackage{url}
\usepackage{fixltx2e} %keeps figure*s in order with figures
\usepackage{color}
\usepackage{rotating}
\newcommand{\comment}[1]{}
% \usepackage{exceltex} % http://www.physik.uni-freiburg.de/~doerr/exceltex/index.en.html
\usepackage[colorlinks=true,linkcolor=blue,citecolor=blue,urlcolor=blue,pdfauthor={Matthew Roughan},pdftitle={The Compleat Catenary}]{hyperref}
\usepackage[table]{xcolor}
\definecolor{palepink}{rgb}{1.0,0.9,0.9}
\rowcolors{2}{white}{palepink}


% http://tex.stackexchange.com/questions/63502/how-to-make-latex-recognize-tick-and-cross-symbols
\usepackage[utf8]{inputenc}
\usepackage{amsfonts}
\usepackage{newunicodechar}
\newunicodechar{✓}{\checkmark}


\setlength{\pdfpagewidth}{8.5 in}
\setlength{\pdfpageheight}{11in}

%\renewcommand{\baselinestretch}{1.01}
% space saving itemise and enumerate environments
\newenvironment{sitemize}{%
  \begin{list}{$\bullet$}{%
    %\setlength{\rightmargin}{\leftmargin}
    \setlength{\itemsep}{0.0cm}%
    \setlength{\leftmargin}{1.5em}%
    \setlength{\topsep}{0cm}%
    \setlength{\parsep}{0mm}}%
  }{\end{list}}

\newenvironment{senumerate}{%
   \begin{list}{\arabic{enumi}.}{%
    \setlength\labelwidth{1.5em}%
    \setlength\leftmargin{1.5em}%
    \setlength{\topsep}{2pt plus 2pt minus 2pt}%
    \setlength\itemsep{0.0cm}%
    \usecounter{enumi}}%
  }{\end{list}}

\newlength{\figurewidth}
\setlength{\figurewidth}{0.9\columnwidth}
\newlength{\captionspace}
\setlength{\captionspace}{-5mm}
\newlength{\figstarspace}
\setlength{\figstarspace}{-4mm}
\newcommand{\ie}{{\em i.e., }}
\newcommand{\eg}{{\em e.g., }}
\newcommand{\etal}{{\em et al.}}
\newcommand{\erdren}{Erd\"{o}s-R\'{e}nyi }
\newcommand{\anonymize}[1]{anonymous}

\renewcommand{\textfraction}{0.01}

\begin{document}
\conferenceinfo{}{}
\numberofauthors{1}
\title{Unravelling Graph Exchange File Formats
  %or \\
  %      The Hitchhikers Guide to Sharing Graph Data
}
\author{Matthew Roughan}
% \date{14 November 2012}
\maketitle
\begin{abstract}
  
\end{abstract}
\category{H.4}{Information Systems Applications}{Miscellaneous}
%A category including the fourth, optional field follows...
\category{D.2.8}{Software Engineering}{Metrics}[graphs, file format]

% \terms{Theory}
% \keywords{ACM proceedings, \LaTeX, text tagging}

% all the formats citations
\nocite{ebert99:_grax,holt06:_gxl,herman02:_graph_drawin,batagelj95:_towar_netml,Kienle:rigi,}

\section{Introduction}

A short search of the Internet revealed that there are well over 60
formats used for storage and exchange of graph data: that is networks
of vertices (nodes, switches, routers, ...) connected by edges (links,
arcs, ...). 

Every new tool for working with graphs seems to come with its own new
graph format. There are reasons for this: new tools are often aimed at
providing a new capability. Sometimes this capability is not supported
by existing formats. And inventing your own new format isn't hard. 

More fundamentally exchange of graph information just hasn't been that
important. Standardised formats for images (and other consumer data)
are crucial for the functioning of digital society. Standardised graph
formats affect a small community of researchers and tool builders. But
this community is growing, and the need for interchange of information
is likewise growing, particularly where the data represent some real
measurements which are hard to collect when and as needed, and so
scientists need to be able to share.

So the current state of affairs is ridiculous. The existing formats do
include many of the features one might need, and some are quite
extensible, so the bottleneck is not the existing formats so much as
information about those formats. This is the gap this monograph aims
to fill.

Many of the formats presented may seem obsolete. Some are quite old
(in computer science years). Some have clearly not survived beyond the
needs of the authors' own pet project. However, we have listed as many
as we could properly document, partially for historical reference, and
partially to show the degree of reinvention in this area. But more
importantly, because old and obscure isn't bad. For instance NetML, a
format that doesn't seem to be used at all by any current toolkits,
incorporates some of the most advanced ideas of any format
presented. A good deal could be learnt by current tool builders if
they were to reread the old documentation on this format.

It is important to note that this paper does not present yet-another
format of our own. It is common, in this and other domains, for the
discussion of previous works to be coloured by the need to justify the
authors' own proposal. Here we aim to be unbiased by the need to
motivate our own toolkit, and so (despite temptation) do not provide
any such.

We do not argue that new graph formats should never be
developed. In some applications new features are needed that are not
present in the existing formats. However, it is critical that those
who wish to propose new ideas should understand whether they are {\em
  really} needed, or whether existing tools provide what they
need. Moreover, in studying the existing formats, and their features,
we learn what should be required in any new format to make it more
than a one-shot, aimed at only one application.



\section{Background}

A mathematical {\em graph} ${\cal G}$ is a set of nodes (or vertices)
${\cal N}$ and edges (or links or arcs) ${\cal E} \subset {\cal N}
\times {\cal N}$.

Graphs (alternatively called networks) have been used for many years
to represent relationships between objects or people. 

There are many subtypes of graphs, and generalisations, some of which
we shall mention below. 

Additional information is often added to a graph: for instance
\begin{itemize}

\item node or link labels (names, types, ...);

\item values (distances, capacity, size, ...); or 

\item routing (paths taken when traversing the graph).

\end{itemize}

It has been necessary for many years for researchers in sociology,
biology, chemistry, computer science, mathematics, statistics and
other areas to be able to store graphs representing concepts as
diverse as state-transition diagrams, computer-software structure,
social networks, biochemical interactions, neural networks, Bayesian
inference networks, geneaologies, computer networks, and many more.
Researchers also need to share data. They have done so by sharing
files. As a result portable file formats for describing graphs have
been around for decades.

This document is concerned with providing information about these
formats, specifically with the intention of moving towards a smaller
number of standard formats (the current trend seems to be progressing
in the other direction).

We only look here at publically disclosed formats, for the obvious
reason that a format can't really be called a data exchange format
unless its definition is public. It is fair to say that although many
were intended for interchange of information, most failed at this and
were only really used for a single tool or database of graphs. In a
few other cases, the format was not intended as an exchange format,
but has become a defacto format by the inclusion of IO routines in
other software than its originator. In any case, we have tried to be
inclusive here: we include anything that might be reasonably called an
exchange format (and which is publically documented to some degree),
rather than trying to exclude those which we guess are not.

\subsection{Related work}

There have been a number of other efforts to gather similar
information by researchers
\cite{bodlaj13:_networ,bernard:_graph_file_format,10:_netwik} and
software distributers \cite{yfiles_io,gephi} but these have generally
collected only a few file formats, and don't provide as much detail on
those considered. However, the results did provide inspiration for
some of the descriptors used here, and do provide some points of view
complementary to those we present here. 
support as advice to users as to which to use 

One additional paper to consider is
\cite{brandes00:_graph_data_format_works_repor}, which was written
specifically with the view of designing a new, more universal graph
format. We deliberately avoid this approach in order to avoid bias in
our discussion.

\section{Descriptors and Discriminators}

In order to describe the formats we will consider here, we need some
simple means to compare and contrast.  Of a necessity, these will
oversimplify some of the issues. For instance, some features that
exist in principle in the format may not be supported in any
software.

What's more, many descriptions of file formats are imprecise. It is
common to describe the format by reference to examples. Although
useful for simple cases, these leave out important details: for
instance: the character set supported, and even more surprisingly, the
format of identifiers. It is often vaguely suggested that these are
numbers, but without formal definition of what is allowed (presumably
non-negative integers, but are numbers outside the 32 bit range
supported?). 

In the following, we make the best estimate of the capabilities of
each format through reference to online documentation, and through a
survey of the file format creators\footnote{For which purpose this
  current document is being used.}. In many cases the results are
inferences, so in this section we will outline the features we
describe, and the assumptions made in compiling our data. However, we
have made the best effort possible to contact authors of formats, and
their comments about capabilities have been given precedence.

There are three main types of descriptors here: 
\begin{description}

\item[file type]: these are simple issues of the type of file storing
  the data: binary vs ASCII, etc.
 
\item[graph types]: this refers to the nature of the graph data that
  can be stored.

\item[attributes]: these are features related to supplemental data
  about nodes and edges, such as labels and values associated with
  these.

\item[general]: this is a grab bag for additional features that don't
  fit in either of the previous classes.

\end{description}
We'll describe each of these in detail below, and then provide a table
of the features vs file formats.

We have tried to include a ``reference date range'' for various
formats, to provide some context for the format. The dates are based
on explicit records from the first note of the file, through to the
last recorded date of maintenance. However, this information is often
not supplied, so we have used, for instance changelogs or copyright
dates on format documentation. In many cases even this is vague, and
so we don't always report a date, or it is open ended. This should not
be seen as a reliable field so much as an attempt to document the
historical development of this field, so much of which is not in the
archival journals. Also important to note is that some are still in
``current'' use, particularly in graph repositories, despite a
narrower range.

For instance, it is interesting to see how XML became flavour of the
day around 2000, while in the most recent past there seem to be
several efforts to design graph formats on top of JSON. 

One last point, this is not intended as a pejorative list. We do not
mean to imply that having a feature is good or bad (though vague
descriptions do seem fairly unhelpful). The aim is to provide
potential users with the background to choose the right format for
their purposes.

\subsection{File Type}

\begin{description}
\item[binary/text]: this is, in principle, a simple distinction in
  file type. However, text files today can use multiple different
  character sets, and this is important because some graphs will be
  labelled with non-English character sets. However, the majority of
  file-format definitions leave unspecified the character set to be
  used. We assume here that the character set is ASCII, unless there
  is some indication otherwise, either an explicit statement, or in
  the case of applications of XML it is assumed that the character set
  supported is Unicode.

\item[representation]: there are quite a few methods to
  represent a graph: 
  \begin{description}
  \item[matrix]: This is simply the graph's full adjacency matrix.

  \item[edge]: This is a list of the graph's edges~\cite{ebert87:_versat_data}.

  \item[smatrix]: The matrix representation is poor for sparse graphs,
    which are common in real situations. However, some tools actually
    store a sparse matrix, which is almost equivalent to an edge
    list. There is a subtle difference in that a matrix view of the
    edges in a network cannot contain much detail about the edges
    (only one number), and so we have a separate name, {\em smatrix},
    for formats that use this type of representation.

  \item[neighbour] This is a list of the graph's nodes, each giving a
    list of neighbours for each node.

  \item[path]: One can also implicitly represent a graph as a series
    of {\em path} descriptions (essentially a path is a list of
    consecutive edges). This could be useful, for instance, with a
    tree or ring. 

    Moreover, graph data is often derived from path data, i.e., a
    series of paths are analysed, and the edges on these become the
    graph. In other cases, one might like to store path information,
    for instance related to routes used on a graph, along with the
    graph.

  \item[constructive]: Graphs can often be described in terms of
    mathematical operations used to construct the graphs: for instance
    graph products on smaller graphs
    \cite{parsonage11:_gener_graph_produc_networ_desig_analy}. See
    \cite{batagelj95:_towar_netml} for a description of ``levels'' of
    graph formats.

    Apart from simple incremental construction, the only format that
    seems to allow this is NetML~\cite{batagelj95:_towar_netml}.

  \item[procedural]: Many graphs can be concisely defined by a
    set of procedures, rather than explicit definition of the nodes
    and links. This type of graph format could be very concise, but
    verges on creating another programming language. In fact, many
    graph libraries for particular programming languages essentially
    provide this, but obviously in a non-portable (between languages)
    way. 

    The only {\em generic} (language independent) format that seems to
    allow this is NetML~\cite{batagelj95:_towar_netml}.

    Any procedural approach admits the possibility of defining a
    method for constructive graph description, but we do not
    automatically count any procedural approach as constructive, unless
    it provides explicit graph-related operations as part of the
    toolkit. 

    % We could, perhaps, also list {\em random} as a subclass of
    % procedural, where the graph (or an ensemble of graphs) is
    % described by some notation specifying a type of random graph,
    % parameters and seeds, but there are no formats (apart from program
    % libraries) that allow this as far as we know.

  \end{description}
  These representations are given varying names in the literature, but
  we use the names above to be clear.

  The representation is important: for a graph with $N$ vertices and
  $E$ edges, the adjacency matrix requires $O(N^2)$ terms, the edge
  list $O(E)$ terms, and the neighbour list $O(N + E)$ terms. However,
  the terms in a matrix are $\{0,1\}$ whereas the terms in the edge
  and neighbour lists are node identifiers (consider they might be 64
  bit integers), so the size of a resulting file based on each
  representation depends on many issues, including the way the data is
  stored in the file. No approach is universally superior.

  Moreover, some may be easier to read and write: for instance a
  neighbour listing may be slightly more compact than an edge list,
  but it has the same number of elements per line, potentially making
  it easier to perform IO in some languages.

  More subtley, a neighbour-list representation treats edges as
  properties of nodes, whereas an edge list treats edges as objects in
  their own right; and the matrix representation treats the graph as
  the only object with nodes and edges as properties of the graph.
  Although a program can internally represent data however it likes,
  and read in a neighbour list into structures that treat edges as
  objects in their own right, the native treatement of data is
  reflected in the ease with which attributes can be added. For
  instance, in a neighbour list it is intrinsically harder to record
  attributes for edges, and in the matrix representation it is harder
  to record attributes for nodes. This is, fundamentally, why we regard
  edge-list and sparse-matrix formats as different.

  Some graph file formats allow alternative representations, and so we
  list all that are possible. However note that this is often actually
  multiple file formats under one name. It seems rare (for obvious
  reasons) to allow a mixed representation.

  We haven't (yet?) reported on whether edge-list formats explictly
  lists nodes or only implicitly lists them as a consequence of
  edges. The latter is briefer, but requires a special case for degree
  0 nodes.
 
  When considering generalisations of graphs, other representations
  are possible (for instance tensors can generalize the concept of an
  adjacency matrix for multilayer networks). However, codification of
  these is an ongoing research topic \cite{kivela:_multil} and so we
  will not try to encapsulate it here.

\item[structure:] we use this field to describe how the file format's
  structure is defined. The cases are: 
  \begin{description}

  \item[simple]: the typical approach to create a graph format is to
    use one line per data item (a node, an edge, or a neighbourhood),
    with the components on a line separated by a standard delineator
    (a comma, tab, or whitespace). There are many variations on this
    theme, some more complex than others, for instance including
    labels, comments or other information. These formats are usually
    specified by a very brief description and one or two
    examples. They rarely specify details such as integer range or
    character set.

  \item[intermediate]: this is a slight advance on a {\em simple} file
    format, in that it includes some grammatical elements. For
    instance, the file may allow definition of new types of labels for
    objects. However, in common with simple files, these are usually
    only specified by a very brief description and one or two
    examples, not a complete grammar.

  \item[BNF]: means that the file format is described using a grammar,
    loosely equivalent to a Backus-Naur Form (BNF). This is perhaps
    the most concise, precise description. When done properly it
    precisely spells out the details of the file in a relatively short
    form.

 % http://en.wikipedia.org/wiki/Backus%E2%80%93Naur_Form

  \item[XML, JSON, SGML, ...]: many file formats extend XML, JSON,
    SGML, or similar generic, extensible file formats. This is a
    natural approach to the problem, and allows a specification as
    precise as BNF, though only through reference to the format being
    extended. Thus it is precise, but sometimes rather difficult to
    ascertain all of the details, unless one is an expert in XML, etc.

    On the other hand, these approaches draw on the wealth of tools
    and knowledge about these data formats. On the other hand again,
    to use those tools the model of your graph object has to map to
    the XML model (or at least be easily transformed into that form).

  \item[Tcl, Lisp, ...]: as noted above one approach to defining a
    graph is procedural. Most of the approaches that allow this are
    extensions or libraries for common programming languages. 

    We will not list every programming language and library as a data
    format though because, generically, such approaches are not
    portable between programming languages. We do mention a few
    formats though (cypher, ns-2 and S-Dot), because translators exist
    from/or to these from other data formats.

  \end{description}

\item[single or multiple files]: most data files are a single file,
  but some formats require multiple files, for instance, a separate
  files for the lists of nodes and edges. Other formats allow
  supplementary information in additional files, so multiple files
  aren't mandatory. We have only classified the files by whether
  multiple files are allowed, not whether they are mandatory (because
  the later requires a distinction about what mandatory would mean:
  does it mean they are required to support basic features or advanced
  features?) 

\item[must be ordered]: most files have some requirement for ordering,
  for instance a header, or tags around data, so we are not concerned
  by that aspect of ordering. We are concerned with whether elements
  of the actual data must be presented in some type of ordering. For
  instance, do nodes have to be defined before we can create an edge
  joining them? This is often unspecified, particularly in simple
  formats, so we draw conclusions from the examples presented. 

  The decision to list a particular file as ordered can seem a little
  arbitrary, but the importance of it is really whether the file can
  be read in one (simple) pass or not. Ordering (e.g., defining nodes
  before using them) can potentially help in a one pass read. This is
  the information we are interested in expressing.

\item[integral meta-data]: meta-data is data about the graph: for
  example, its name, its author, the date created, and so on. This is
  very important data, but many formats provide no means to include it
  in the file, and instead rely on external records. We refer to
  meta-data as {\em integral} if it is contained in the file itself.  

  Some formats allow meta-data through unstructured comments. This is
  better than nothing, but lack of structure of the comments means
  these are not machine readable, in general.

  Some file formats provide only a limited range of meta-data fields,
  whereas others are arbitrarily extensible. To distinguish the
  various cases we fill this field with one of the following:
  \begin{description}
  \item[no]: No meta-data is allowed.
  \item[comments]: Unstructured meta-data is allowed in comments.
  \item[fixed]: A defined set of meta-data can be included.
  \item[arbitrary]: An explicit mechanism is described allow allow
    arbitrary meta-data to be included.
  \end{description}

\item[built-in compression]: it is easy enough to compress a graph-file
  using common utilities such as gzip, and typical compression ratio
  will be reasonably good as graph files often have many repeated
  strings. However, one format provides for compression of the graph
  as it is written, in much the way image file formats allow intrinsic
  compression of the image. Such an approach requires a
  graph-compression algorithm, though two other formats provided some
  crude mechanisms to reduce the size of the file.

  % Two factors should be considered for such compression approaches:
  % (i) the space saved, and (ii) the time required for
  % compression/decompression.

\end{description}

\subsection{Graph Types}

\begin{description}
\item[directed/undirected]: The two basic forms of graph are the
  directed and undirected graph. In the former edges (or arcs) imply a
  relation from one node to another. In the later an edge implies a
  relationship in both directions.

  Some graph formats specify one or the other; others allow the user
  to specify which; and the most general allow the user to specify
  directed (or not) for each edge. 

  The difficulty with this is that many graph formats fail to specify
  anything. We assume that, in the absence of explicit statements to
  the contrary, a graph format is {\em directed} if the edges/arcs are
  specified by source/target or from/to, or some other directional
  nomenclature. We also assume that matrix formats are directed unless
  there is specific mention of mechanism to represent the upper
  triangular part of the matrix alone.

  Finally, in one case, the format is explicitly restricted to DAGs
  (Directed Acyclic Graphs). 

\item[multi-graph]: a multi-graph is a graph generalisation that
  allows (i) self-loops, and (ii) more than one edge between a single
  pair of nodes.

  Some formats specifically allow, or disallow multi-graphs. A few
  allow loops, but not multi-edges. Many, however, say nothing on the
  topic. We assume in this case that formats presenting either matrix
  or neighbour representations don't allow multi-graphs. It is
  technically possible to represent a multi-graph in these cases, but
  this would require special processing of the information, and unless
  we see an indication this is present we assume it is not. Edge
  lists, however, can easily cope with multi-graphs. We suspect it is
  left to the software supporting the data format to make a decision
  about how to deal with these cases, and the decision may be
  inconsistent between supporting software. Hence it seems important
  that when an edge-based representation leaves the question
  unspecified, we say this.

\item[hyper-graphs]: a hyper-graph allows edges that connect more than
  two nodes. These are useful for some problems: for instance
  indicating a multi-access medium in a computer network (such as a
  wireless network).

  Support for hyper-graphs requires specialised data structures for
  hyper-edges, so unless a format explicitly states it can support
  these and presents the mechanism we assume it cannot.

  Alternatively, one can realise hyperedges by adding a new type of
  node, and creating simple edges from this to all the hyper-edge
  adjacencies. Obviously, this simplifies in one respect, but
  complicates in another.  

\item[hierarchy]: it is common for graphs to have sub-structure, for
  instance nodes that themselves contain graphs.

  Several formats provide mechanisms to record this
  substructure. Unfortunately, there does not seem to be a
  consistently used definition of this type of structure
  \cite{bildhauer11:_dhhtg}, and so we see differences not just in the
  representation, but also what exactly is being represented. The
  problem becomes even more complicated when hierachy and hypergraphs
  are combined \cite{kivela:_multil} (there is at least one proposed
  solution \cite{bildhauer11:_dhhtg} but it does not seem to be widely
  used yet). 

  We don't try to list this level of detail here though, we simply
  note whether the format provides this feature.

\item[meta-graph]: a meta-graph~\cite{basu07:_metag_applic} is a
  generalisation of a graph, multi-graph, hyper-graph, and
  hierarchical graph. As far as we know, no format yet supports
  meta-graphs\footnote{Note the term ``meta-graph'' is somewhat
    overloaded, e.g., there is at least one package called {\em
      metagraph} that has nothing to do with the mathematical
    meta-graph.}, but this is included as a feature as an indication
  of the type of feature that might require a new format, or extended
  version of an existing format.

\item[edge-edge links]: generally, a graph creates links between
  nodes, but we could generalise the concept to allow meta-edges that
  join edges as well (this is different from a meta-graph).

\end{description}

\subsection{Attributes}

\begin{description}

\item[edge weights]: a very common requirement is to store a numerical
  value associated with an edge. Generically, we call this a
  weight. Many formats provide the facility to keep one such value.

\item[multiple attributes]: Some formats allow one to keep multiple
  labels (numerical or otherwise) for each node and/or edge. 

  For some formats these are fixed (e.g., they allow a name and a
  value), whereas others allow arbitrary lists of attributes. We don't
  yet distinguish these two cases. 

\item[default values]: specifying the value of a weight or attributed
  for every edge or node can be laborious (if it has to be done by
  hand), and wasteful of space. Moreover, it makes it hard to see
  structure in the data. Simply providing a default value for the
  common case can improve the situation. We include here the case of
  simple inheritance of values through a tree of ``class'' structures
  on the objects. For instance, nodes can be given a type which
  conveys a default value to be overridden by a more specific type or
  particular value. Notice here we are no speaking of inheritance
  through the graph itself, but a structure on top of the graph.

\item[multiple inheritance]: a few formats allow values to be derived
  through multiple inheritance of values from multiple classes the
  belong to. Thus allowing a node to have, for instance, a type
  ``router'' which conveys that it is an Internet router, with
  appropriate characteristics for such a device, from vendor ``Cisco''
  which appropriate characteristics for that vendor.
  
  Once again, inheritance is not through the structure of the graph,
  but through a further structure defined on the graph objects. 

\item[visualisation data]: files that allow multiple (extensible)
  attributes can always provide data to be used in visualising the
  graph, but here we refer to formats that explicitly provide such
  data. It needs to be explicit because it has a particular use in
  software, different from the use of more arbitrary associated data
  such as labels. 

  The level of visualisation data varies dramatically: some formats
  only allow position information for nodes, whereas others allow SVG
  definitions to be used in drawing the nodes. Still others provide
  guidance about which layout algorithms to use in displaying the
  graph. 

  There is not space here to document all of the variations possible,
  so we simply indicate whether any such data is defined or not. 

\item[ports]: this is a specialised piece of layout information: often
  ports are specified by a compass direction, and indicate where on a
  node the link should join to it. We include it separate to the
  previous field because port-based information can also carry
  semantic information about the relationship between links on a
  complex node: e.g., the arrangement of links on a real device like
  an Internet router. 

\item[temporal data/dynamics]: a topic of interest for many years is
  analysis/visualisation of graphs as they
  change~\cite{ebert87:_versat_data}. One way to store this
  information is as a series of ``snap-shot'' graphs, but storing it
  all together in the same file has some appeal. A few formats provide
  some variant on this: allowing links or nodes to be given a
  lifetime, or proving ``edits'' to the graph at specific epochs.

\end{description}

\subsection{General}

\begin{description}
\item[extensible]: some formats allow extensibility in varying
  forms. We only consider them to have this facility, however, if they
  provide an explicit mechanism. For instance, we do not regard all XML
  derivatives as intrinsically extensible because they could, in
  principle, be extended using standard XML techniques. The format has
  to explain the explicit mechanism whereby it is extended.

  Simply adding extra attributes is not considered extensibility.

\item[schema checking]: a format that provides an explicit mechanism
  to check that a file is in a valid format is useful. We only say it
  has this facility if a tool exists to perform the check (a
  schema-checking program, DTD, or other similar formal tool).

\item[checksums]: It is possible for large data files to become
  corrupted. A common preventative (or at least check for this
  problem) is to use a checksum. This is possible for all files, but
  we say that a given format has this capability if it includes it as
  a internal component (usually checking everything except the
  checksum itself). Only a few formats contain this check.

\item[external data references]: Some formats allow reference to
  external files. This could be for visualisation data, meta-data, or
  other purposes. There are several approaches and views on external
  references, but essentially the idea is to record whether it is
  expected that all relevant information will be in the file, or
  whether there might be something external.  Again, we look for an
  explicit explanation of the mechanism, not a generic belief that it
  is inherited from the parent file format.

\item[multiple graphs]: Some formats allow multiple graphs to be held
  in one file. Again, we only count this as a feature if the
  specification explains how explicitly.

\item[incremental specification]: A small number of formats that
  present multiple graphs allow these graphs to be specified
  incrementally. This is subtly different from including temporal
  dynamics, as there is no implication of time, and the different
  graphs could potentially be unrelated (for instance, this might be
  used to describe graph edit distance problems).

  In a sense incremental specification is a simple case of
  constructive graph definition, but it is a very limited case, with
  specific application, so we list it separately.

\end{description}


\section{The File Formats}

As noted the aim here is to describe graph {\em exchange} formats, \ie
formats that are used to exhange data between scientists and
programming environments. Not all of the formats started out that way
-- some were intended as intern formats for a particular software
system, but have become {\em de facto} exchange formats when another
system sought to leverage existing data by incorporating an existing
format. A few of the formats are still primarily internal to a single
system, but are important to describe because they exhibit an
interesting feature, but in the main we concentrate on those that were
designed with data exchange in mind, or have been used in that way in
practice. 

This list is incomplete. There are some formats that we have seen
mentioned, but been unable to find documentation for (e.g.,
Gem2Ddraw). There are undoubtedly others that we have missed.

Moreover there are a few that we have lumped together under the
general heading of {\em TGF} or {\em Adj} because they are all
functionally equivalent simply delimited edge lists. There is no point
in listing every variant of this approach: there are many and they
vary mainly on the choice of storage (plain ascii -- Excel), and
delimiter (tabs and commas are common).

Moreover, we have deliberately omitted generic file formats that
could, in principle, contain a graph: e.g., XML, JSON, SGML, RDF,
Avro, YAML, and so on, unless there is a specific extension of these
designed to provide support for graphs, in which case we list the
specific not the generic. For instance, several software tools say
that they can read/write JSON or other generic serialisations of data,
but without details of exactly what is being serialised, then these
are not useful interchange formats.

We also aim to avoid, for simple practicality, formats that represent
data that has a graph structure, but whose main content is not the
graph. For instance the HTML WWW: the graph structure of this is
vastly smaller than the content and HTML is intended to store both in
a distributed fashion. If one wished to represent the graph of the
WWW, then another format seems indicated. Other examples include SBML
(the Systems Biology Markup Language), and FOAF \cite{foaf}. For that
matter, formats such as image formats could, in principle, contain an
adjacency matrix, but we shall omit these here. 

{\bf The attached spreadsheet contains the currently known formats,
  and a first draft of features for these formats. It is to be
  considered unverified and potentially highly inaccurate at this
  point.}

Tables~\ref{tab:formats} through \ref{tab:other} provide the
information about the formats.

Additional formats are welcome, but we plan to add only those which
are aimed at portable data exchange, not internal data formats for use
only within one tool (unless the internal format illustrates a
particular issue very well).

% \begin{tabular}{r|ll}
%   \inctab{graphs.xls!FormatList!A2!C2}
%   \hline
%   \inctab{graphs.xls!FormatList!A3!C82}
% \end{tabular}
% \inccell{graphs.xls!FormatList!A7}

\hyphenpenalty=100000

\begin{table*}[p]
  \centering
  {\small
    \begin{tabular}{rrrllp{20mm}}
      % TABLE{../graph_format_big_list.xlsx}{excel2latex/tab0.tex}{1}{2-2}{ & {\bf {!B}} & & & {\bf {!E}} & {\bf {!I}} \\} 
      \input{excel2latex/tab0.tex}
      \hline
      % TABLE{../graph_format_big_list.xlsx}{excel2latex/tab1.tex}{1}{3-*}{{!A} &  \href{!G}{!B} & \cite{!F} & {!D} & {!E} & {!I} \\} 
      \input{excel2latex/tab1.tex} 
    \end{tabular} 
  }
  \caption{The format list.}
  \label{tab:formats}
\end{table*}


\begin{table*}[p]
  \centering
  {\small
    \begin{tabular}{r|llllp{12mm}p{15mm}p{17mm}}
      % TABLE{../graph_format_big_list.xlsx}{excel2latex/tab2.tex}{1}{2-2}{{\bf {!B}} & {\bf {!K}} & {\bf {!L}} & {\bf {!M}} & {\bf {!N}} & {\bf {!O}} & {\bf {!P}} & {\bf {!Q}} \\} 
      \input{excel2latex/tab2.tex}
      \hline
     % TABLE{../graph_format_big_list.xlsx}{excel2latex/tab3.tex}{1}{3-*}{{!B} & {!K} & {!L} & {!M} & {!N} & {!O} & {!P} & {!Q} \\}
      \input{excel2latex/tab3.tex} 
    \end{tabular} 
  }
  \caption{File types.}
  \label{tab:file_types}
\end{table*}


\begin{table*}[p]
  \centering
  {\small
    \begin{tabular}{r|llllll}
      % TABLE{../graph_format_big_list.xlsx}{excel2latex/tab4.tex}{1}{2-2}{{\bf {!B}} & {\bf {!S}} & {\bf {!T}} & {\bf {!U}} & {\bf {!V}} & {\bf {!W}} & {\bf {!X}} \\} 
      \input{excel2latex/tab4.tex}
      \hline
     % TABLE{../graph_format_big_list.xlsx}{excel2latex/tab5.tex}{1}{3-*}{{!B} & {!S} & {!T} & {!U} & {!V} & {!W} & {!X} \\}
      \input{excel2latex/tab5.tex} 
    \end{tabular} 
  }
  \caption{Graph types.}
  \label{tab:graph_types}
\end{table*}


\begin{table*}[p]
  \centering
  {\small
    \begin{tabular}{r|lllp{20mm}p{20mm}l}
      % TABLE{../graph_format_big_list.xlsx}{excel2latex/tab6.tex}{1}{2-2}{{\bf {!B}} & {\bf {!Z}} & {\bf {!AA}} & {\bf {!AB}} & {\bf {!AC}} & {\bf {!AD}} & {\bf {!AE}} \\} 
      \input{excel2latex/tab6.tex}
      \hline
     % TABLE{../graph_format_big_list.xlsx}{excel2latex/tab7.tex}{1}{3-*}{{!B} & {!Z} & {!AA} & {!AB} & {!AC} & {!AD} & {!AE}  \\}
      \input{excel2latex/tab7.tex} 
    \end{tabular} 
  }
  \caption{Allowed attributes.}
  \label{tab:attributes}
\end{table*}


\begin{table*}[p]
  \centering
  {\small
    \begin{tabular}{r|lp{15mm}lp{20mm}p{15mm}p{20mm}}
      % TABLE{../graph_format_big_list.xlsx}{excel2latex/tab8.tex}{1}{2-2}{{\bf {!B}} & {\bf {!AH}} & {\bf {!AI}} & {\bf {!AJ}} & {\bf {!AK}} & {\bf {!AL}}  & {\bf {!AM}}\\} 
      \input{excel2latex/tab8.tex}
      \hline
     % TABLE{../graph_format_big_list.xlsx}{excel2latex/tab9.tex}{1}{3-*}{{!B} & {!AH} & {!AI} & {!AJ} & {!AK} & {!AL} & {!AM} \\}
      \input{excel2latex/tab9.tex} 
    \end{tabular} 
  }
  \caption{Other properties.}
  \label{tab:other}
\end{table*}


\clearpage
\section{Decisions}

The list above is not intended to be pejorative. However, it is
inevitable that potential users need to make decisions about which
format to use. There are several issues that need be considered in
such a decision, and although the first is the feature list required,
there are others:
\begin{description}

\item[data size]: the size of the graph data to be recorded and used
  is an important factor in file format decisions. This is sometimes
  glossed over when XML-style formats are considered: these are very
  redundant formats, and hence much larger than needed, but they
  compress well. Hence, the compressed version may be no longer than a
  tighter initial specification. However, the issue of read/write time
  (and indeed compression/decompression time) still depends greatly on
  the format's wordiness. Large graphs need tighter formats: either
  binary formats, or at least those that avoid unnecessary bloat. 

  On the far end of the spectrum is the possibility of graph-specific
  compression being part of the storage process (much as many image
  formats provide image compression as an integral features). Only one
  format we found provides true graph-based compression: BVGraph. 

\item[edge density]: edge density affects the choice of best
  representation of a graph. Very sparse graphs are best represented
  by edge lists, moderately sparse graphs are (perhaps) slightly
  better stored as neighbour lists, and dense graphs may be better
  stored as a full adjacency matrix. 

\item[access method]: most graph formats are designed to be read
  serially directly into memory in their entirety. None we saw
  provided support for random (or indexed subgraph) access to part of
  a graph. Few formats seem to specifically address this, though some
  public databases of graph data provide interfaces for querying
  components of the graph.

  In other cases, a single graph might be part of a larger database,
  and this seems only to have been addressed by ad-hoc mechanisms. 

\item[human readability]: Implicitly we need the file to be machine
  readable, but a file that is more easily comprehended by humans is
  potentially better because it is easier to enter and check. This
  might seem strange to consider, but many of graph examples were
  entered at least in part by hand. Human readability requires a text
  file in a logical format, but it also needs to avoid: (i) bloat,
  which distracts the reader with unnecessary text, and (ii) the file
  to be organised neatly. XML formats often fail on these: the first
  because of the volume of tags, and the second because they allow
  organisations which are unreadable, e.g., with all the text on one
  line.

  Ultimately, human readability is a highly subjective criteria. Some
  people may find XML easy to read, and others get distracted by the
  tags. As such, we won't comment on it further here.

\item[documentation]: Through compiling the information used in this
  paper it has become obvious that a key limitation of many formats is
  incomplete documentation. Hidden assumptions, specification by
  (limited) examples, and/or documentation by source code are all
  common. Ideally, any truly portable format should have a complete,
  highly-specific schema; human readable documentation (with
  examples); and source code. All of these together provide the
  ideal documentation.

\item[support]: Finally, the support for the format in a variety of
  tools is a crucial requirement for exchange of data. Likewise,
  support for formats in a variety of public databases makes it more
  useful. We shall consider this issue in more detail below.   

\end{description}

\subsection{Software Support} 

The most difficult issue surrounding software support is that with
vague specifications, a piece of software may notionally support a
file format, and yet still be incompatible with other software
notionally supporting the same format.

For instance, software might
\begin{itemize}
\item be able or unable to cope with multi-graphs;
\item fail to accept integers outside a particular range;
\item have varying case sensitivity; 
\item be unable to read the right character set;
\item be unable to read strings beyond a particular length (very few
  formats specify buffer or string lengths); or
\item fail to cope with files larger than some size. 
\end{itemize}
Size is interesting, because almost no documentation exists for size
limits for any data formats. However, it should be reasonably obvious
that if 32 bit integers are used, then the largest number of (integer)
identifiers is around 4 billion. In the past this was large enough
that the need to specify it may have seemed small. With today's
graphs, this could be an important limitation.

Even more pernicious is partial support for a format. Even when
documented this makes our job hard, but partial support is not often
documented. Instances include:
\begin{itemize}
\item hyper-graphs supported in the format, but not in software;  or
\item some small number of formats make mention of allowing complex
  numbers; or
\item partial support for hierarchy (i.e., the file can be read, but
  the subgraph structure is not retained).
\end{itemize}

Even more complex is the fact that some features may be supported on
read or write, but not both. 

% Generally, software is designed for a particular purpose, and when a
% feature falls outside that purpose, the software often fails to
% support it.

% Our goal here is not to criticise software, only to provide some
% guidance over the level of support for a format, and the visibility
% this format has in the general graph-research community. So instead of
% determining the exact details of support for each format, in each
% piece of software, we will report the software authors' claims of
% support only. The goal is not to report technical detail so much as
% the level of awareness and interest a format engenders. 

% {\bf See the attached spreadsheet for a list of software and the
%   supported formats.}

% , but is still a tiny cross-section of the
% graph toolkits provided today. We have sampled primarily on the
% results of simple web searches, but also when a format was created for
% a particular piece of software, we have included that software. 

The list of potential software is long, even more so than the list of
formats, so we won't try to survey them here as well. Instead we refer
readers to \cite{bodlaj13:_networ}, which contains a cross-section of
both formats and their software support. 

A common conclusion amonfst those who look at this type of data is
that GraphML and Pajek are the most commonly supported in modern
systems, but they are by no means universal. 

\subsection{Public DB support}

The other type of support we might wish to see is general support
amongst those who provide data publicly. There are many public
databases that provide example networks for benchmarking or
research. We provide a list in \autoref{tab:pdb} of some of the better
known of these with their format choices. Addition data sources are
listed in \cite{alhajj14:_sourc_networ_data}, and a detailed taxonomy
and examples of computer-network data appears in
\cite{battista13:_handb_graph_drawin_visual}.

\begin{table*}[p]
  \centering
  {\small
    \begin{tabular}{r|ll}
      % TABLE{../graph_format_big_list.xlsx}{excel2latex/tab10.tex}{3}{2-2}{{\bf {!A}} & {\bf {!C}} & {\bf {!B}} \\} 
      \input{excel2latex/tab10.tex}
      \hline
     % TABLE{../graph_format_big_list.xlsx}{excel2latex/tab11.tex}{3}{3-*}{\href{!D}{!A} & {!C} & {!B} \\}
      \input{excel2latex/tab11.tex} 
    \end{tabular} 
  } 
  \caption{Public Databases.}
  \label{tab:pdb}
\end{table*}

There is clear winner here: slightly prefered is a simple adjaceny
list due to its least-common-denominator status (but note that this
isn't really one format, so much as a collection of equivalent
formats). Overall, however, the formats seem to be written for the
data rather than the other way around. That, in itself, is an
illustration of part of the problem here.

\subsection{Discussion}

The point of all this: what should be done here, how should one
proceed. There are three major considerations:
\begin{itemize}
\item what representation of a graph (or generalised graph) will be
  used: edge or neighbour list, adjacency matrix, paths, or some
  constructive or procedural approach;  
\item what additional information is to be added, and how flexible
  this information should be; and
\item what encapsulation of the data is to be used (XML and more
  recently JSON see to be favourites). 
\end{itemize}

Compression seems to be an issue on the horizon, particularly as
graphs get larger. Is there an equivalent to Huffman coding for
graphs, can we compress matrices using Lempel-Ziv as in images, what
type of ...
 
\section{To Do}

Evaluate criteria given, and expand on a couple which could use more
detail (e.g., extensible, vs static meta-data).

Check details again.

Other features/descriptors we might like to include:

\begin{description}

\item[self-describing]: this format data isn't filled in yet (as it is
  yet another rather rubbery term when dealing with imprecise
  formats), but it refers to whether a file provides its own
  definition of its format.

\item[distribution]: most graph formats are monolithic in that the
  entire graph is held in one file. Even those that allow multiple
  files use this to structure the type of information each contains,
  not to spread the information evenly.

  As graph data becomes larger, and the need to query subsections of
  the graph grows, we need to be able to create modularity in the
  graph representation. Formats that provide the ability to distribute
  the graph information over multiple (indexed) files provides a
  capability that could be very useful \cite{bildhauer11:_dhhtg}.

\item[node list]: does the format have a separate node list, or is
  this implicit in the edges.

\item[multi-layer]: graphs that have a layer
  structure~\cite{kivela:_multil} (resembling in some cases hierarchy,
  and in some cases temporal evolution, but more flexible than either
  by itself). Multilayer graphs can naturally be described by
  adjacency tensors.

\end{description}



\subsection{Stats of the results}

how many fall into each category (with mixed as one)

  representation
      binary
      ascii
      other text
      text or binary

  ...
 
multi-layer or multi-relational	networks	


\section{Conclusion}

The science of graphs and networks needs portable, well-documented,
precisely defined, exchange formats. There are many existing formats,
and this paper seeks to unravel this mess, most notably with the aim
of reducing the number of new formats developed.

One size probably does not fit all though.  There is a clear need for
at least three major types of file format:
\begin{itemize}

\item a general, flexible, extensible approach such as GraphML;

\item a quick and dirty approach that satisfies the least common
  denominator for the exchange of information to/from the simplest
  software; and

\item a very efficient (compressed) format for very large graphs. 

\end{itemize}

Its not clear that any format at present has a complete enough list of
features to take the roll of the first format. No doubt this will
continue to evolve as well, as new features are required.

The second is easy, but there are very many contenders, and settling
on one will be hard.

The final one should be seen as an interesting research topic. 

Maybe what is needed is actually a container format: allowing
specification of parts of a graph in alternative formats. Or allowing
specification of meta-data and labels in an XML-like format, but the
edge data in a more compact form. 


% \section*{Acknowledgements}

% Many people have helped improve the quality of information in this
% paper, specifically: 
% \begin{itemize}
% \item PROGRES: Andy Sch\"{u}rr,
% \item SNAP: Jure Leskovec and Rok Sosi\v{c},
% \item GraphLab: Danny Bickson, Sethu Raman, Haijie Gu,
% \end{itemize}



{\small  
\bibliographystyle{IEEEtran}
\bibliography{topology,ip_traffic}{}
}

% \appendix
% \input{GAdetails.tex}

\end{document}
