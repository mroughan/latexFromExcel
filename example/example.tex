\documentclass{article}
\usepackage{amsmath}
\usepackage{amssymb}
\usepackage{graphicx}
\usepackage{subfig}
% \DeclareCaptionType{copyrightbox}
\usepackage{cite}
\usepackage{url}
\usepackage{fixltx2e} %keeps figure*s in order with figures
\usepackage{color}
\usepackage{rotating}
\newcommand{\comment}[1]{}
\usepackage[colorlinks=true,linkcolor=blue,citecolor=blue,urlcolor=blue,pdfauthor={Matthew Roughan},pdftitle={The Compleat Catenary}]{hyperref}
\usepackage[table]{xcolor}
\definecolor{palepink}{rgb}{1.0,0.9,0.9}
\rowcolors{2}{white}{palepink}

% http://tex.stackexchange.com/questions/63502/how-to-make-latex-recognize-tick-and-cross-symbols
\usepackage[utf8]{inputenc}
\usepackage{amsfonts}
\usepackage{newunicodechar}
\newunicodechar{✓}{\checkmark}
\usepackage{verbatim}


\begin{document}

\section{Introduction}

A quick example of how to call the latexFromExcel script. 

\begin{table*}[h]
  \centering
  {\small
    \begin{tabular}{r|ll}
      % latexFromExcel{example.xlsx}{table_header.tex}{1}{2-2}{{\bf {!B}} & {\bf {!C}} & {\bf {!D}} \\}
      % printf FILE "{\\bf {%s}} & {\\bf {%s}} & {\\bf {%s}} \\\\ \n", ($cs{Bm},$cs{Cm},$cs{Dm})
{\bf {Name}} & {\bf {Comment}} & {\bf {Current}} \\ 

      \hline
      % latexFromExcel{example.xlsx}{table_body.tex}{1}{3-*}{  \href{!E}{!B} & {!C} & {!D} \\}
      % printf FILE "  \\href{%s}{%s} & {%s} & {%s} \\\\ \n", ($cs{Em},$cs{Bm},$cs{Cm},$cs{Dm})
  \href{https://github.com/mroughan/latexFromExcel}{John} & {Hello world} & {\checkmark} \\ 
  {Adam} & {Not today} & {\checkmark} \\ 
  {Reg} & {One} & {} \\ 
  \href{https://github.com/mroughan/latexFromExcel}{Belinda} & {Two} & {} \\ 
  {Laura} & {Three} & {} \\ 
  {Darryl} & {} & {} \\ 
  {Ashley} & {} & {} \\ 
  {Madison} & {Here we are now} & {\checkmark} \\ 
 
    \end{tabular} 
  }
  \caption{Example table from example.xlsx.}
  \label{tab:attributes}
\end{table*}

\noindent The actual tables generated are repeated here:

\verbatiminput{table_body.tex}

\noindent Note the first line gives the print statement generating each line of
the file, and then it is followed by one line for each (included) row
of the table in that format, but after filtering through the
following:

\verbatiminput{filters.txt}


\end{document}
